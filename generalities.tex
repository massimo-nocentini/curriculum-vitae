\documentclass[twoside,openright,titlepage,fleqn,
headinclude,11pt,a4paper,BCOR5mm,footinclude ]{scrbook}
%--------------------------------------------------------------
        \newcommand{\myTitle}{Analisi di reti metaboliche basata su
          propriet\`a di connessione\xspace}
% use the right myDegree option
\newcommand{\myDegree}{Corso di Laurea in Informatica\xspace}
%\newcommand{\myDegree}{
	%Corso di Laurea Specialistica in Scienze e Tecnologie 
	%dell'Informazione\xspace}
\newcommand{\myName}{Massimo Nocentini\xspace}
\newcommand{\myProf}{Pierluigi Crescenzi\xspace}
\newcommand{\myOtherProf}{Nome Cognome\xspace}
\newcommand{\mySupervisor}{Nome Cognome\xspace}
\newcommand{\myFaculty}{
	Facolt\`a di Scienze Matematiche, Fisiche e Naturali\xspace}
\newcommand{\myDepartment}{
	Dipartimento di Sistemi e Informatica\xspace}
\newcommand{\myUni}{\protect{
	Universit\`a degli Studi di Firenze}\xspace}
\newcommand{\myLocation}{Firenze\xspace}
\newcommand{\myTime}{Anno Accademico 2010-2011\xspace}
\newcommand{\myVersion}{Version 0.1\xspace}
%--------------------------------------------------------------
\usepackage[latin1]{inputenc} 
\usepackage[T1]{fontenc} 
\usepackage[square,numbers]{natbib} 
\usepackage[fleqn]{amsmath}  
\usepackage[italian]{babel}
%--------------------------------------------------------------
\usepackage{dia-classicthesis-ldpkg} 
%--------------------------------------------------------------
% Options for classicthesis.sty:
% tocaligned eulerchapternumbers drafting linedheaders 
% listsseparated subfig nochapters beramono eulermath parts 
% minionpro pdfspacing
\usepackage[eulerchapternumbers,subfig,beramono,eulermath,
	parts]{classicthesis}
%--------------------------------------------------------------
\newlength{\abcd} % for ab..z string length calculation
% how all the floats will be aligned
\newcommand{\myfloatalign}{\centering} 
\setlength{\extrarowheight}{3pt} % increase table row height
\captionsetup{format=hang,font=small}
%--------------------------------------------------------------
% Layout setting
%--------------------------------------------------------------
\usepackage{geometry}
\geometry{
	a4paper,
	ignoremp,
	bindingoffset = 1cm, 
	textwidth     = 13.5cm,
	textheight    = 21.5cm,
	lmargin       = 3.5cm, % left margin
	tmargin       = 4cm    % top margin 
}
%--------------------------------------------------------------
\usepackage{listings}
\usepackage{hyperref}
% My Theorem
\newtheorem{oss}{Observation}[section]
\newtheorem{exercise}{Exercise}[section]
\newtheorem{thm}{Theorem}[section]
\newtheorem{cor}[thm]{Corollary}

\newtheorem{lem}[thm]{Lemma}

\newcommand{\vect}[1]{\boldsymbol{#1}}

% questo comando e' relativo alle correzioni che puo
% apportare il prof se lo desidera.
\newcommand{\prof}[1]{\boldsymbol{#1}}

% instead of boldsymbol I can use the arrow above the letter with
%\newcommand{\vect}[1]{\vec{#1}}

% page settings
% \pagestyle{headings}
%--------------------------------------------------------------
\begin{document}
\frenchspacing
\raggedbottom
\pagenumbering{roman}
\pagestyle{plain}
%--------------------------------------------------------------
% Frontmatter
%--------------------------------------------------------------
%\include{titlePage}
\pagestyle{scrheadings}
%--------------------------------------------------------------
% Mainmatter
%--------------------------------------------------------------
\pagenumbering{arabic}

% settings for the lstlisting environment
\lstset{
	language = java
	, numbers = left 
	, basicstyle=\sffamily%\footnotesize
	%, frame=single
	, tabsize=2
	, captionpos=b
	, breaklines=true
	, showspaces=false
	, showstringspaces=false
}

\chapter*{Generalit\`a}
\begin{description}
\item[Nome] Massimo
\item[Cognome] Nocentini
\item[Nazionalit\`a] Italiana
\item[Data Nascita] $08/01/1986$
\item[Luogo Nascita] Firenze
\item[Codice Fiscale] $NCNMSM86A08D612R$
\item[Residenza] Via Ronco Corto, n. $98$, CAP $50143$, 
        Firenze, Italia
\item[Recapito telefonico] $3201162059$
\item[Recapito elettronico] \url{massimo.nocentini@gmail.com} 
\item[Titolo Studio Conseguito] FACOLTA' DI SCIENZE MATEMATICHE FISICHE E NATURALI, Laurea in INFORMATICA 
\item[Data conseguimento Laurea] $22/02/2012$
\item[Sede conseguimento Laurea] Firenze
\item[Votazione] $110$
\end{description}


\end{document}
