% cv example for classicthesis.sty
%\documentclass{scrartcl}
\documentclass[a4paper]{tufte-handout}
%\usepackage[LabelsAligned]{currvita} % nice cv style
\usepackage{url}
\usepackage{inputenc}
\usepackage{fontenc}
%\usepackage{textcomp}
%\usepackage{newtxtext} % for text
\usepackage{graphicx}
\usepackage{float}
\usepackage{amsmath}
\usepackage{amsthm}
\usepackage{amssymb}
\usepackage{eulervm}
\usepackage{stmaryrd}
\usepackage{bbold}

% solved in https://github.com/Tufte-LaTeX/tufte-latex/issues/64
\renewcommand\allcapsspacing[1]{{\addfontfeature{LetterSpace=15}#1}}
\renewcommand\smallcapsspacing[1]{{\addfontfeature{LetterSpace=10}#1}}

\setmonofont[Scale=0.8]{Menlo}
\hypersetup{colorlinks}

\begin{document}

    \title{Curriculum Vit\ae\,et Studiorum}

    \author{Massimo Nocentini\newline
        \small{{massimo.nocentini@gmail.com}},
        \small{{massimo.nocentini@unifi.it}}\newline
        \small{{https://github.com/massimo-nocentini}}
    }

    \maketitle

    \section{Personal data}
    
    I was born in Italy on 8 January 1986 and I live in Florence, Italy. 

    \section{Education}

    \begin{itemize}
        \item PhD degree in Computer Science, thesis title \emph{An algebraic
        and combinatorial study of some infinite sequences of numbers supported
        by symbolic and logic computation}, affiliated to the Dipartimento di
        Statistica, Informatica e Applicazioni (DiSIA~
        \sidenote{\url{https://www.disia.unifi.it/}}) and advised by prof.
        Donatella Merlini\sidenote{\url{http://local.disia.unifi.it/merlini/}}, University of Florence, from 2015 to 2019.
        \item Master Laurea degree in Computer Science, thesis title \emph{Patterns in Riordan arrays}, 
            supervised by prof. Donatella Merlini, University of Florence, 2015.
        \item Laurea degree in Computer Science, thesis title \emph{Analysis of metabolic networks based on connection properties}, 
            supervised by prof. Pierluigi Crescenzi, University of Florence, 2012.
        \item Maturity exam on Computer Science, Meucci Technical Institute, ABACUS project, Florence, 2005.
    \end{itemize}

    \section{Scientific activity}
    
    His research activity concerns (i)~the study of \textit{formal methods
    and their applications} to the analysis of algorithms and data structures,
    (ii)~supporting them with \textit{software abstractions} implemented using functional
    programming languages, (iii)~toward the field of \textit{mechanized mathematics}.  
    A solid base for such methods comes from the field
    of analytic combinatorics, which comprises tools such as generating
    functions, Riordan arrays and the symbolic method. Many interesting books
    by Flajolet and Sedgewick\sidenote{Flajolet and Sedgewick.
    \emph{Analytic Combinatorics}, Cambridge University Press, 2009.},
    Knuth\sidenote{Knuth. \emph{The Art of Computer Programming}, vol.  1-3,
    Addison-Wesley, 1973.} and Graham et al. \sidenote{Graham, Knuth and
    Patashnik. \emph{Concrete Mathematics: A Foundation for Computer Science},
    Addison-Wesley, 1994.} exist on those topics; moreover,  see
    Harrison\sidenote{Harrison. \textit{Handbook of Practical Logic and
    Automated Reasoning}, Cambridge University Press, 2009; and \textit{The HOL
    Light theorem prover}, User manual, 2017.}, Friedman and Felleisen
    \sidenote{Friedman and Felleisen. \textit{The Little Schemer} and
    \textit{The Seasoned Schemer} and \textit{The Little MLer}, MIT Press.} and
    Byrd et al.\sidenote{Byrd, Friedman and Kiselyov. \textit{The Reasoned
    Schemer}, MIT Press.} for implementation aspects. 

    He desires to have a solid grasp of such powerful techniques in order to
    think about combinatorial \emph{interpretations} of
    analytic results about classes of abstract objects in order to show
    \emph{combinatorial meanings} and, possibly, characterizations in terms
    of lattice paths, urn models, bracelet configurations, boards
    tiling and so on, in the spirit of Benjamin and Quinn \sidenote{Benjamin
    and Quinn. \emph{Proofs that really counts}, Mathematical Association of
    America, 2003.} and Stanley\sidenote{Stanley. \emph{Enumerative
    combinatorics. {V}ol. 1\&2}, Cambridge University Press.}.

    He believes that abstract and formal contexts should be paired up with
    sounding computer programs that show their beauty and elegance; this
    parallel path allows him to code in Lisp, Python, OCaml, Smalltalk and
    Haskell during his daily work.

    His PhD had been supervised by prof \textit{Donatella Merlini} where he
    studied advanced topic about \textit{Riordan arrays} such as binary words
    avoiding patterns, lattice paths enumeration problems and transformations
    of infinite sequences of numbers; he has deepened his understanding of
    those concepts in order to make them central in his PhD thesis. At the same
    time, he proposed many prototypes using the Python language, in order to
    get introduced to asynchronous programming, terminal-oriented interfaces,
    symbolic computations and relational programming; by the way, he believe
    that those areas are fundamentals and preparatory to tackle more complex
    and demanding problems, machine learning and data mining in particular.

    Moreover, he is collaborating with prof \textit{Marco Maggesi}
    \footnote{\url{http://web.math.unifi.it/users/maggesi/}} to enhance the HOL
    Light theorem prover with an extension of the goals and tactics mechanism
    to support the relational paradigm, in the spirit of
    $\mu$kanren\footnote{Hemann and Friedman. \textit{$\mu$Kanren: a Minimal Functional
    Core for Relational Programming}, Scheme2013, Alexandria.}.

    \section{Papers}

    \begin{itemize}

        \item Donatella Merlini, Massimo Nocentini. \emph{Functions and Jordan canonical forms of Riordan matrices},
        in \textit{Linear Algebra and its Applications}, Volume 565, 15 March 2019, Pages 177-207.

        \item Donatella Merlini, Massimo Nocentini. \emph{Crawling, (pretty) printing and graphing the OEIS}, 
        working paper\footnote{\url{http://local.disia.unifi.it/wp_disia/2018/wp_disia_2018_06.pdf}} at DiSIA.

        \item Donatella Merlini, Massimo Nocentini. \emph{Algebraic generating functions for languages
            avoiding Riordan patterns}, in \textit{Journal of Integer Sequences}, Volume 21, Article 18.1.3, 2018.

        \item Donatella Merlini, Massimo Nocentini. \emph{Colouring Catalan triangle}.

    \end{itemize}
    
    \section{Conferences}

    \begin{itemize}
        \item \emph{ESUG}, September 2018, Cagliari, Italy \sidenote{\url{https://esug.github.io/2018-Conference/conf2018.html}}: volunteer student and
        contributed the talk \textit{Relational Programming in Smalltalk} \sidenote{\url{https://github.com/massimo-nocentini/microkanrenst/releases/download/v1.0/esug.pdf}}.
        \item \emph{<Programming>}, April 2018, Nice, France \sidenote{\url{https://2018.programming-conference.org}}: participant.
        \item \emph{ICFP}, September 2017, Oxford, UK \sidenote{\url{https://conf.researchr.org/home/icfp-2017}}: volunteer student.
        \item \emph{EuroPython}, July 2017, Rimini, Italy \sidenote{\url{https://ep2017.europython.eu/}}: participant.
        \item \emph{ECOOP}, July 2016, Rome, Italy \sidenote{\url{http://2016.ecoop.org/}}: volunteer student.
        \item \emph{Second International Symposium on Riordan Arrays and Related Topics}, 
            July 2015 Lecco, Italy \sidenote{\url{https://www.mate.polimi.it/RART2015/}}: contributed a talk about modular Catalan triangle $\mathcal{C}_{\equiv_{2}}$.
    \end{itemize}

    \section{Seminars and Schools}

    \begin{itemize}
        \item \textit{PhD defense}, University of Florence \sidenote{\url{https://github.com/massimo-nocentini/massimo-nocentini.github.io/releases/download/PhD-defense/talk.pdf}}.
        \item \textit{summary of $3$rd year} PhD activities, University of Florence \sidenote{\url{https://github.com/massimo-nocentini/massimo-nocentini.github.io/releases/download/v1.0/talk.pdf}}.
        \item \textit{Logic and Relational Programming} at Logic Department, University of Florence \sidenote{\url{http://massimo-nocentini.github.io/PhD/mkpy/talk.html\#}}.
        \item \textit{summary of $2$nd year} PhD activities, University of Florence \sidenote{\url{http://massimo-nocentini.github.io/PhD/second-year-summary/talk.html\#}}.
        \item \textit{Algebraic gf avoiding Riordan patterns} at AORC Open School, Sungkyunkwan University \sidenote{\url{http://shb.skku.edu/_custom/skk/_common/board/download.jsp?attach_no=29038}}.
        \item \textit{EOIS tools} at AORC Open School, Sungkyunkwan University \sidenote{\url{http://massimo-nocentini.github.io/PhD/skku-aorc-2017/oeistools.html\#}}.
        \item \textit{summary of $1$st year} PhD activities, University of Florence \sidenote{\url{http://massimo-nocentini.github.io/PhD/first-year-summary/talk.html\#}}.
    \end{itemize}

    \section{Teaching}

    He did two classes about \emph{SymPy} to introduce symbolic manipulations
    on top of the Python language, within a course on \emph{Analysis of
    Algorithms} taught by Donatella Merlini at the University of Florence; in
    addition, he translated lab sessions code from Maple to Python collected in
    notebooks available online
    \sidenote{\url{https://github.com/massimo-nocentini/pacc/tree/master/paa-course}}.

    \section{Github}

    Recently he joined the \textit{Square Bracket Associates} organization; moreover,
    he pushes into the following repos:
    \begin{description}
        \item[https://github.com/massimo-nocentini/on-scheme] holds stuff about
        the Scheme dialect of Lisp. In particular, here he explores
        \textit{continuations, $\mu$Kanren, union-find structures, lazy streams
        memoization} and \textit{abstract computing devices (Landin's SECD machine)}.

        \item[https://github.com/massimo-nocentini/microkanrenst] holds his
        Smalltalk implementation of $\mu$Kanren, shown at ESUG2018; precisely,
        it provides \textit{triangular substitutions, complete (but unfair)
        solutions enumeration} and \textit{structural induction by heavy double
        dispatchings}.

        \item[https://github.com/massimo-nocentini/Booklet-microKanren] holds
        docs and explanations of the relational paradigm provided in the repo before,
        using Pillar as writing environment, just started.

        \item[https://github.com/massimo-nocentini/kanren-light] holds a
        parallel goals and tactics mechanism for Harrison's \textit{HOL Light}
        theorem prover that is inspired by $\mu$Kanren, co-authored with prof
        Marco Maggesi.

        \item[https://github.com/massimo-nocentini/simulation-methods] holds
        the implementation of the framework of matrices functions, namely the
        process that lifts a scalar function to a matrix function using
        eigenvalues and generalized Lagrange bases.

        \item[https://github.com/massimo-nocentini/oeis-tools] holds a suite of tools
        to mine the Online Encyclopedia of Integer Sequences, providing a crawler,
        a pretty printer and a grapher of relations among sequences of numbers; implemented
        in Python, allows him to play with \verb|async| and \verb|await| primitives.

        \item[https://github.com/massimo-nocentini/competitive-programming]
        holds his solutions to a few problems from the UVa only judge;
        moreover, some jupyter notebooks are provided about bitmasking,
        backtracking, tilings, Gray codes, recursively-defined structures and
        dynamic programming.

        \item[https://github.com/massimo-nocentini/recurrences-unfolding] holds
        a Python implementation of a framework that allows arbitrary unfoldings
        of recurrence relations, such as the famous $f_{n+2}=f_{n+1}+f_{n}$;
        precisely, the main idea is to use each rec relation as a rewriting
        rule that can be used in pattern matching to instantiate vars in
        subscripts. 

        \item[https://github.com/massimo-nocentini/on-python] holds study
        material to explore features provided by the Python language;
        precisely, he takes into account coroutines, metaprogramming, a double
        dispatcher and collects some pointers.

        \item[https://github.com/massimo-nocentini/microkanrenpy] a Pythonic
        implementation of $\mu$Kanren, providing \textit{complete} and
        \textit{fair} solutions enumerations and \textit{impure logical
        operators}, paired up with a decent doc and a test suite that covers
        \textit{all} questions of \textit{The Reasoned Schemer} book.

        \item[https://github.com/massimo-nocentini/master-thesis] holds the
        \TeX\, sources and implementations of his Master Thesis defended at the
        University of Florence; precisely, he proposes another characterization
        of Riordan arrays that exposes their $A$-sequence's generating
        functions for the Bell subgroup.

        \item[https://github.com/massimo-nocentini/cagd] holds Julia and Python
        implementations of de Casteljau algorithm, Bezier curves, BSplines,
        tensor product and triangular Bezier patches; he worked on this to
        fullfil the  course of Computer Aided Graphic Design at University of
        Florence, given by prof Alessandra Sestini and prof Costanza Conti.

        \item[https://github.com/massimo-nocentini/theory-of-programming-languages]
        holds a study of type inference according to Benjamin Pierce
        \sidenote{Pierce. \textit{Types and Programming Languages}, MIT Press.} and
        the corresponding implementation using SML/NJ.

        \item[https://github.com/massimo-nocentini/on-the-little-schemer] holds
        study material and translation of \textit{The Little Schemer} using the
        \textit{CommonLisp} language in order to understand higher-order
        programming and play with \verb|lisp-unit|; moreover, a Java
        implementation of the \verb|Y| combinator is given.

        \item[https://github.com/massimo-nocentini/reasoning-about-little-books]
        holds study material using the SML language, some derivation of the \verb|Y|
        combinator are given (even an older one \sidenote{Friedman and Felleisen.
        \textit{The Little LISPer}, MIT Press.}) and most definitions from
        the "little books" had been reworked.

        \item[https://github.com/massimo-nocentini/network-reasoner] holds a
        simulator of gas distribution networks, according to various
        parameters; joint work with engineer Fabio Tarani and implemented in
        C\# on the Mono platform.

        \item[https://github.com/massimo-nocentini/chicken-zmq] holds an
        incomplete set of bindings of the distributed messaging framework
        ZeroMQ\footnote{\url{http://zeromq.org/}} for the Chicken Scheme
        system. This translation aims to be bare minimal and a one-to-one
        porting of the original definitions and it is stressed against
        the examples of the official
        guide\footnote{\url{http://zguide.zeromq.org/page:all}}.

        
    \end{description}

    \section{Working activity}

    During his studies he worked in middle-sized software houses
    \sidenote{Formerly at \url{https://www.commitsoftware.it/}, lately at
    \url{http://www.negens.com/site/home.html}.} developing mainly client-server
    applications using industrial-strength languages such as Java and C\#, for
    about eight years, part-time relationships in parallel with his studies; in short,
    \begin{itemize}
        \item a software to generate bet systems for the Italian circuit
        \textit{Sisal}, written in C\#; precisely, the software allows a group
        of lotteries to authenticate using a secure channels and personal
        tokens with a server that regularly fetched odds from the Sisal
        provider. Then, it generates reports with tables containing a
        subset of events that satisfies the lotteries' constraints.
        \item a software that takes into account VAT registers and economic
        transactions for small bussnesses. Again, written in C\# interfacing 
        with MSSQL servers, using advanced SQL features about pivot tables,
        cursors, views and stored procedures; this software were developed in
        a joint effort with a group in Milan, Italy.
        \item little projects about sensors, microcontrollers and home
        automation using Linux boxes and the RaspberryPi, interfacing with C
        code while scripting in Python; heavy use of shell and network protocols
        for messages exchange.
    \end{itemize}
        
    \vfill
    \noindent\makebox[\linewidth]{\rule{\textwidth}{0.4pt}}
    For Italian readers -- autorizzo al trattamento dei dati personali secondo
    quanto previsto dalla legge numero 196/03.
    
\end{document}
