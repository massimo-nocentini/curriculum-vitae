% cv example for classicthesis.sty
\documentclass[10pt,a4paper]{scrartcl}
%\usepackage[LabelsAligned]{currvita} % nice cv style
\usepackage{url}
\usepackage{concrete}
\usepackage[T1]{fontenc}
\usepackage{graphicx}
\usepackage{float}
\usepackage{amsmath}
\usepackage{amsthm}
\usepackage{amssymb}
\usepackage{caption}
\usepackage{stmaryrd}
\usepackage{bbold}


\begin{document}

    \title{Curriculum Vit\ae et Studiorum}

    \author{Massimo Nocentini\\ 
        \small{Dipartimento di Statistica, Informatica e Applicazioni}\\
        \small{50134, Florence, Italy}\\
        %\small{\url{https://github.com/massimo-nocentini}}\\
        \small{\url{massimo.nocentini@unifi.it}}
        }

    \maketitle

    \section{Personal informations}
    
    I was born in Italy, 8 January 1986 and live in Florence, Via Ronco Corto 98,
    50143, Italy. Currently I'm a PhD student at the University of Florence, 
    Dipartimento di Statistica, Informatica e Applicazioni (DiSIA), working
    with prof. Donatella Merlini.

    \section{Education}
    \begin{itemize}
        \item maturit\`a
        \item laurea magistrale
    \end{itemize}

    \section{Scientific activity}
    
    His research activity is mainly concerned with the study of 
    formal methods and their application to the analysis of algorithms and data structures, 
    supported by implementations written using symbolic and functional languages.
    A solid base for such methods comes from the field of analytic combinatorics, 
    which comprises tools such as generating functions, Riordan arrays and the symbolic method. Many
    interesting books by Flajolet and Sedgewick\footnote{P. Flajolet and R.
    Sedgewick, \emph{Analytic Combinatorics}, Cambridge University Press, 2009.},
    Knuth\footnote{D. Knuth, \emph{The Art of Computer Programming}, vol.  1-3,
    Addison-Wesley, 1973.} and Graham et al. \footnote{Graham, Knuth and Patashnik,
    \emph{Concrete Mathematics: A Foundation for Computer Science}, Addison-Wesley,
    1994} exist on those topics. 


    He desires to have a solid grasp of such powerful techniques in order to
    think about combinatorial \emph{interpretations} of
    analytic results about classes of abstract objects in order to show
    \emph{combinatorial meanings} and, possibly, characterizations in terms
    of lattice paths, urn models, bracelet configurations, boards
    tiling and so on, in the spirit of Benjamin and Quinn \footnote{Benjamin
    and Quinn, \emph{Proofs that really counts}, Mathematical Association of
    America, 2003} and Stanley\footnote{Richard Stanley, \emph{Enumerative
    combinatorics. {V}ol. 1\&2}, Cambridge University Press}. Moreover, he wants
    to apply such techniques and interpretations to the analysis of algorithms
    as far as the analytic aspect is concerned, and to data structures for the
    combinatorial one. 

    He believes that all this abstract and formal context should be paired up with
    sounding computer programs to show the beauty and elegance of such topics;
    this parallel path allows him to enhance and deepen his knowledge in functional
    and symbolic programming, using languages like Lisp, Python and Haskell in his daily work.
    For this reason, during the first year of his PhD, he wrote a bunch of Jupyter 
    notebooks about Gray codes, backtracking algorithms applied to tiling problems
    and the generation of recursive structures, and finally, application
    of bit-masking techniques to speed up symbolic computations\footnote{\url{https://github.com/massimo-nocentini/competitive-programming/tree/master/tutorials}};
    he publishes his works as open-source projects on GitHub\footnote{\url{https://github.com/massimo-nocentini/}}.

    At the same time, he continues to work on Riordan Arrays, studied in his
    master thesis\footnote{\textit{Patterns in Riordan Arrays}, Massimo Nocentini, October 2015, University of Florence}, 
    focusing on new characterizations to spot properties of their
    structure. One example is the $h$-characterization $\mathcal{R}_{h(t)}$ of a
    Riordan array $\mathcal{R}$, developed and explored within the thesis.
    Another path that he is following is the study of two important objects, $A$-sequence
    $\lbrace a_{n}\rbrace_{n\in\mathbb{N}}$ and $A$-matrix $\lbrace
    a_{ij}\rbrace_{i,j\in\mathbb{N}}$ respectively, generalizing them in order to
    discover new combinatorial identities. This approach is supported by a framework
    written using the Python language that performs unfolding of recurrence relations
    from the symbolic point of view: this is a work in progress, yet ready
    to be stressed against relations of general interest\footnote{\url{https://github.com/massimo-nocentini/recurrences-unfolding}}.


    The other topic of his thesis shows his interest in the description and formalization of Riordan
    arrays under the light of modular arithmetic. He has shown congruences
    about \emph{Pascal} array $\mathcal{P}$ and its inverse $\mathcal{P}^{-1}$. He
    has also proved a formal characterization for the \emph{Catalan} array
    $\mathcal{C}$. These results were presented in a talk contributed at a recent
    conference held in Lecco\footnote{Second International Symposium on Riordan
    Arrays and Related Topics, RART$2015$} and is the topic of a submitted paper.

    Currently, he is working with prof. Donatella Merlini\footnote{\url{donatella.merlini@unifi.it}}
    on advanced topic involving Riordan arrays: in particular, on binary words avoiding patterns, 
    lattice paths enumeration problems and transformations of infinite sequences of numbers.
    He would deepen his understanding of such topics in order to make them central in his PhD thesis;
    moreover, he is designing a symbolic framework to implement a subset of most important and useful
    definitions taken from literature on this field.
    
    \iffalse % % working activity section, I don't know if it is helpful in this kind of cv {{{

    \section{Working activity}

    During his studies he worked in middle-size software houses developing mainly client-server
    applications using industrial-strength languages such as Java and C\#, for about eight years.
    \fi
    % }}}
        
    
\end{document}
