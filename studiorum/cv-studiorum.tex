% cv example for classicthesis.sty
%\documentclass{scrartcl}
\documentclass[a4paper]{tufte-handout}
%\usepackage[LabelsAligned]{currvita} % nice cv style
\usepackage{url}
\usepackage{inputenc}
\usepackage{fontenc}
%\usepackage{textcomp}
%\usepackage{newtxtext} % for text
\usepackage{graphicx}
\usepackage{float}
\usepackage{amsmath}
\usepackage{amsthm}
\usepackage{amssymb}
\usepackage{eulervm}
\usepackage{stmaryrd}
\usepackage{bbold}

% solved in https://github.com/Tufte-LaTeX/tufte-latex/issues/64
\renewcommand\allcapsspacing[1]{{\addfontfeature{LetterSpace=15}#1}}
\renewcommand\smallcapsspacing[1]{{\addfontfeature{LetterSpace=10}#1}}

\setmonofont[Scale=0.8]{Menlo}
\hypersetup{colorlinks}

\begin{document}

    \title{Curriculum Vit\ae\,et Studiorum}

    \author{Massimo Nocentini\newline
        \small{{massimo.nocentini@gmail.com}},
        \small{{massimo.nocentini@unifi.it}}\newline
        \small{{https://github.com/massimo-nocentini}}
    }

    \maketitle

    \section{Personal data}
    
    I was born in Italy on 8 January 1986 and I live in Florence, Italy. 

    \section{Education}

    Currently I'm a PhD student at the University of Florence, affiliated to the
    Dipartimento di Statistica, Informatica e Applicazioni (DiSIA~
    \sidenote{\url{https://www.disia.unifi.it/}}), supervised by prof
    \textit{Donatella Merlini}
    \sidenote{\url{http://local.disia.unifi.it/merlini/}}; in the past,

    \begin{itemize}
        \item Master Laurea degree in Computer Science, thesis title \emph{Patterns in Riordan arrays}, 
            supervised by prof. Donatella Merlini, University of Florence, 2015.
        \item Laurea degree in Computer Science, thesis title \emph{Analysis of metabolic networks based on connection properties}, 
            supervised by prof. Pierluigi Crescenzi, University of Florence, 2012.
        \item Maturity exam on Computer Science, Meucci Technical Institute, ABACUS project, Florence, 2005.
    \end{itemize}

    \section{Scientific activity}
    
    His research activity concerns (i)~the study of \textit{formal methods
    and their applications} to the analysis of algorithms and data structures,
    (ii)~supporting them with \textit{software abstractions} implemented using functional
    programming languages, (iii)~toward the field of \textit{mechanized mathematics}.  
    A solid base for such methods comes from the field
    of analytic combinatorics, which comprises tools such as generating
    functions, Riordan arrays and the symbolic method. Many interesting books
    by Flajolet and Sedgewick\sidenote{Flajolet and Sedgewick.
    \emph{Analytic Combinatorics}, Cambridge University Press, 2009.},
    Knuth\sidenote{Knuth. \emph{The Art of Computer Programming}, vol.  1-3,
    Addison-Wesley, 1973.} and Graham et al. \sidenote{Graham, Knuth and
    Patashnik. \emph{Concrete Mathematics: A Foundation for Computer Science},
    Addison-Wesley, 1994.} exist on those topics; moreover,  see
    Harrison\sidenote{Harrison. \textit{Handbook of Practical Logic and
    Automated Reasoning}, Cambridge University Press, 2009; and \textit{The HOL
    Light theorem prover}, User manual, 2017.}, Friedman and Felleisen
    \sidenote{Friedman and Felleisen. \textit{The Little Schemer} and
    \textit{The Seasoned Schemer} and \textit{The Little MLer}, MIT Press.} and
    Byrd et al.\sidenote{Byrd, Friedman and Kiselyov. \textit{The Reasoned
    Schemer}, MIT Press.} for implementation aspects. 


    He desires to have a solid grasp of such powerful techniques in order to
    think about combinatorial \emph{interpretations} of
    analytic results about classes of abstract objects in order to show
    \emph{combinatorial meanings} and, possibly, characterizations in terms
    of lattice paths, urn models, bracelet configurations, boards
    tiling and so on, in the spirit of Benjamin and Quinn \sidenote{Benjamin
    and Quinn. \emph{Proofs that really counts}, Mathematical Association of
    America, 2003.} and Stanley\sidenote{Stanley. \emph{Enumerative
    combinatorics. {V}ol. 1\&2}, Cambridge University Press.}.

    He believes that abstract and formal contexts should be paired up with
    sounding computer programs that show their beauty and elegance; this
    parallel path allows him to code in Lisp, Python, OCaml and Haskell during
    his daily work.

    Currently, he is supervised and collaborates with prof \textit{Donatella
    Merlini} on advanced topic about \textit{Riordan arrays}, in particular on
    binary words avoiding patterns, lattice paths enumeration problems and
    transformations of infinite sequences of numbers; he would deepen his
    understanding of those concepts in order to make them central in his PhD
    thesis. 

    Moreover, he is collaborating with prof \textit{Marco Maggesi}
    \footnote{\url{http://web.math.unifi.it/users/maggesi/}} to enhance the HOL
    Light theorem prover with an extension of the goals and tactics mechanism
    to support the relational paradigm, in the spirit of
    $\mu$kanren\footnote{Hemann and Friedman. \textit{$\mu$Kanren: a Minimal Functional
    Core for Relational Programming}, Scheme2013, Alexandria.}.

    \section{Papers}

    \begin{itemize}

        \item Donatella Merlini, Massimo Nocentini. \emph{Functions and Jordan canonical forms of Riordan matrices},
        currently under review by the journal \textit{Linear Algebra and its Applications}, 2018.

        \item Donatella Merlini, Massimo Nocentini. \emph{Algebraic generating functions for languages
            avoiding Riordan patterns}, in \textit{Journal of Integer Sequences}, Volume 21, Article 18.1.3, 2018.

        \item Donatella Merlini, Massimo Nocentini. \emph{Colouring Catalan triangle}.

    \end{itemize}
    
    \iffalse
        \item \emph{Recurrence unfolding}\sidenote{\url{https://github.com/massimo-nocentini/recurrences-unfolding}}: 
            We provide a framework, written using the Python language
            on top of \texttt{SymPy} module, to perform arbitrary unfolding of recurrence relations. The main idea
            is to consider a set of possibly mutually defined recurrence relations, call it $\Omega$, and use each
            one of them as a \emph{rewriting rule} in the sense of using the left-hand side (``lhs'' for short) 
            as a term to be matched in order to instantiate the right-hand side (``rhs'', respectively) accordingly; 
            finally, use the new rhs as a replacement for the term that starts the matching. 

            Unlike "plain" substitution, we perform an extended matching strategy on the lhs, 
            allowing the mathematician to write relations that include a coefficient in the lhs: so a term in the rhs
            matches successful if it is possible to find a substitution for free variables that makes equal both
            the indexed symbol and the coefficient.

            To the time of this document, we have a working prototype for relations that involve indexed terms of the
            form $f_{n_{1}, \ldots, n_{k}}$ for desired $k\in\mathbb{N}$, with the constraint that the recurrence relation 
            use constant coefficients. We're working to fully handle arbitrary recurrence relations. For the sake of 
            clarity, we show, first, an application to the sequence of Fibonacci numbers\sidenote{\url{http://nbviewer.jupyter.org/github/massimo-nocentini/recurrences-unfolding/blob/master/notebooks/fibonacci-numbers-unary-indexed-unfolding.ipynb}},
            according to the unary-indexed recurrence $f_{n+2}=f_{n+1}+f_{n}$; second, an application to the Pascal array
            \sidenote{\url{http://nbviewer.jupyter.org/github/massimo-nocentini/recurrences-unfolding/blob/master/notebooks/pascal-array-doubly-indexed-unfolding.ipynb}},
            according to the doubly-indexed recurrence $d_{n+1,k+1} = d_{n,k}+d_{n,k+1}$.

            We aim to show possibly new or hard to recognize identities over classes of combinatorial objects counted by
            relations under study; therefore, this prototype could be seen as an helper for the mathematician to understand
            how a recurrence behaves doing unfolding, leaving to him/her the analytic check of spotted patterns seen 
            while unfolding the recurrence. In preparation.

        \item \emph{Riordan Arrays}: Following ideas of unfolding recurrence relations, we consider matrices in the Riordan group,
            leaving them completely symbolical, focusing only on their $A$-sequences which dictates relations among coefficients.
            We choose a row $r$ and start to unfold coefficients from row $r+1$ to the last, in order to build a matrix that 
            depends on a small set of coefficients, namely those lying on the former $r+1$ rows. This allow us to rewrite
            the original matrix as a sum of ``inner'' matrices, all of them sharing the same relation among coefficients.
            We're working on a prototype that built such splitting, collecting material in a notebook
            \sidenote{\url{http://nbviewer.jupyter.org/github/massimo-nocentini/recurrences-unfolding/blob/master/notebooks/matrix-recurrences-unfolding.ipynb}}.
            In preparation.

        \item\emph{OEIS mining}: We develop a \emph{web-crawler} targeting the OEIS \sidenote{\url{http://oeis.org/}}, in order
            to fetch sequences according to the \emph{cross-references} section included in search results. Collected 
            informations can be represented as a graph, where each node represent a sequence and there is a reference
            between two sequences if they appears in the cited section. Such graph can be manipulated in order to remove
            not interesting sequences for the study of interest, in order to apply graph mining techniques. This is an
            ongoing project, however we have a first application\sidenote{\url{http://nbviewer.jupyter.org/github/massimo-nocentini/competitive-programming/blob/master/tutorials/oeis-mining.ipynb?flush_cache=true}}. In preparation.

        \item\emph{OEIS search result pretty printer}: On top of the web-crawler defined in the previous topic, 
            we provide a bunch of Python functions to implement a
            \emph{pretty printer} for search result returned by OEIS. Use Jupyter notebook as working
            environments, we provide an API in order to query the OEIS by either sequence identifier or
            segment or open content, parsing results as json documents and returning an IPython display object
            that shows them directly in your notebook. This approach has the following ideas: simple API to 
            perform a search, filtering all sections provided by the OEIS, list and table representation 
            for sequences that have one and two dimensions, respectively. We believe that having an integrated
            environment, without neither switching between browser tabs nor put references to external contents, 
            allows us to build reproducible, self-contained notebooks. We provide a tutorial\sidenote{\url{http://nbviewer.jupyter.org/github/massimo-nocentini/competitive-programming/blob/master/tutorials/oeis-interaction.ipynb?flush_cache=true}}. In preparation.
    \fi
    
    \section{Conferences}

    \begin{itemize}
        \item \emph{ESUG}, September 2018, Cagliari, Italy \sidenote{\url{https://esug.github.io/2018-Conference/conf2018.html}}: volunteer student and
        contributed the talk \textit{Relational Programming in Smalltalk} \sidenote{\url{https://github.com/massimo-nocentini/microkanrenst/releases/download/v1.0/esug.pdf}}.
        \item \emph{ICFP}, September 2017, Oxford, UK \sidenote{\url{https://conf.researchr.org/home/icfp-2017}}: volunteer student.
        \item \emph{EuroPython}, July 2017, Rimini, Italy \sidenote{\url{https://ep2017.europython.eu/}}: participant.
        \item \emph{ECOOP}, July 2016, Rome, Italy \sidenote{\url{http://2016.ecoop.org/}}: volunteer student.
        \item \emph{Second International Symposium on Riordan Arrays and Related Topics}, 
            July 2015 Lecco, Italy \sidenote{\url{https://www.mate.polimi.it/RART2015/}}: contributed a talk about modular Catalan triangle $\mathcal{C}_{\equiv_{2}}$.
    \end{itemize}

    \section{Seminars and Schools}

    \begin{itemize}
        \item \textit{Logic and Relational Programming} at Logic Department, University of Florence \sidenote{\url{http://massimo-nocentini.github.io/PhD/mkpy/talk.html\#}}.
        \item \textit{summary of $2$nd year} PhD activities, University of Florence \sidenote{\url{http://massimo-nocentini.github.io/PhD/second-year-summary/talk.html\#}}.
        \item \textit{Algebraic gf avoiding Riordan patterns} at AORC Open School, Sungkyunkwan University \sidenote{\url{http://shb.skku.edu/_custom/skk/_common/board/download.jsp?attach_no=29038}}.
        \item \textit{EOIS tools} at AORC Open School, Sungkyunkwan University \sidenote{\url{http://massimo-nocentini.github.io/PhD/skku-aorc-2017/oeistools.html\#}}.
        \item \textit{summary of $1$st year} PhD activities, University of Florence \sidenote{\url{http://massimo-nocentini.github.io/PhD/first-year-summary/talk.html\#}}.
    \end{itemize}

    \section{Teaching}

    He did two classes about \emph{SymPy} to introduce symbolic manipulations
    on top of the Python language, within a course on \emph{Analysis of
    Algorithms} taught by Donatella Merlini at the University of Florence; in
    addition, he translated lab sessions code from Maple to Python collected in
    notebooks available online
    \sidenote{\url{https://github.com/massimo-nocentini/pacc/tree/master/paa-course}}.

    \section{Github}

    \section{Working activity}

    During his studies he worked in middle-sized software houses
    \sidenote{\url{https://www.commitsoftware.it/} and
    \url{http://www.negens.com/site/home.html}} developing mainly client-server
    applications using industrial-strength languages such as Java and C\#, for
    about eight years, part-time relationships in parallel with his studies.
        
    
\end{document}
