% cv example for classicthesis.sty
\documentclass[10pt,a4paper]{scrartcl}
%\usepackage[LabelsAligned]{currvita} % nice cv style
\usepackage{url}
\usepackage{concrete}
\usepackage[T1]{fontenc}
\usepackage{graphicx}
\usepackage{float}
\usepackage{amsmath}
\usepackage{amsthm}
\usepackage{amssymb}
\usepackage{caption}
\usepackage{stmaryrd}
\usepackage{bbold}


\begin{document}

    \title{Curriculum Vit\ae\,et Studiorum}

    \author{Massimo Nocentini\\ 
        \small{Dipartimento di Statistica, Informatica e Applicazioni}\\
        \small{50134, Florence, Italy}\\
        \small{\url{massimo.nocentini@unifi.it}}\\
        \small{\url{https://github.com/massimo-nocentini}}\\
        }

    \maketitle

    \section{Personal data}
    
    Born in Italy, 8 January 1986; living in Via Ronco Corto 98, Florence 50143, Italy. 

    \section{Education}

    Currently I'm a PhD student at the University of Florence, 
    Dipartimento di Statistica, Informatica e Applicazioni (DiSIA), working
    with prof. Donatella Merlini\footnote{\url{http://local.disia.unifi.it/merlini/}}.

    \begin{itemize}
        \item Master Laurea degree in Computer Science, thesis title \emph{Patterns in Riordan arrays}, 
            supervised by prof. Donatella Merlini, University of Florence, 2015.
        \item Laurea degree in Computer Science, thesis title \emph{Analysis of metabolic networks based on connection properties}, 
            supervised by prof. Pierluigi Crescenzi, University of Florence, 2012.
        \item Maturity exam on Computer Science, Meucci Technical Institute, ABACUS project, Florence, 2005.
    \end{itemize}

    \section{Scientific activity}
    
    His research activity is mainly concerned with the study of 
    formal methods and their application to the analysis of algorithms and data structures, 
    supported by implementations written using symbolic and functional languages.
    A solid base for such methods comes from the field of analytic combinatorics, 
    which comprises tools such as generating functions, Riordan arrays and the symbolic method. Many
    interesting books by Flajolet and Sedgewick\footnote{P. Flajolet and R.
    Sedgewick, \emph{Analytic Combinatorics}, Cambridge University Press, 2009.},
    Knuth\footnote{D. Knuth, \emph{The Art of Computer Programming}, vol.  1-3,
    Addison-Wesley, 1973.} and Graham et al. \footnote{Graham, Knuth and Patashnik,
    \emph{Concrete Mathematics: A Foundation for Computer Science}, Addison-Wesley,
    1994} exist on those topics. 


    He desires to have a solid grasp of such powerful techniques in order to
    think about combinatorial \emph{interpretations} of
    analytic results about classes of abstract objects in order to show
    \emph{combinatorial meanings} and, possibly, characterizations in terms
    of lattice paths, urn models, bracelet configurations, boards
    tiling and so on, in the spirit of Benjamin and Quinn \footnote{Benjamin
    and Quinn, \emph{Proofs that really counts}, Mathematical Association of
    America, 2003} and Stanley\footnote{Richard Stanley, \emph{Enumerative
    combinatorics. {V}ol. 1\&2}, Cambridge University Press}. Moreover, he wants
    to apply such techniques and interpretations to the analysis of algorithms
    as far as the analytic aspect is concerned, and to data structures for the
    combinatorial one. 

    He believes that all this abstract and formal context should be paired up with
    sounding computer programs to show the beauty and elegance of such topics;
    this parallel path allows him to enhance and deepen his knowledge in functional
    and symbolic programming, using languages like Lisp, Python and Haskell in his daily work.
    For this reason, during the first year of his PhD, he wrote a bunch of Jupyter 
    notebooks about Gray codes, backtracking algorithms applied to tiling problems
    and the generation of recursive structures, and finally, application
    of bit-masking techniques to speed up symbolic computations\footnote{\url{https://github.com/massimo-nocentini/competitive-programming/tree/master/tutorials}}.

    At the same time, he continues to work on Riordan Arrays, studied in his
    master thesis\footnote{Massimo Nocentini, \textit{Patterns in Riordan Arrays}, October 2015, University of Florence}, 
    focusing on new characterizations to spot properties of their
    structure. One example is the $h$-characterization $\mathcal{R}_{h(t)}$ of a
    Riordan array $\mathcal{R}$, developed and explored within the thesis.
    Another path that he is following is the study of two important objects, $A$-sequence
    $\lbrace a_{n}\rbrace_{n\in\mathbb{N}}$ and $A$-matrix $\lbrace
    a_{ij}\rbrace_{i,j\in\mathbb{N}}$ respectively, generalizing them in order to
    discover new combinatorial identities. This approach is supported by a framework
    written using the Python language that performs unfolding of recurrence relations
    from the symbolic point of view: this is a work in progress, yet ready
    to be stressed against relations of general interest.


    The other topic of his thesis shows his interest in the description and formalization of Riordan
    arrays under the light of modular arithmetic. He has shown congruences
    about \emph{Pascal} array $\mathcal{P}$ and its inverse $\mathcal{P}^{-1}$. He
    has also proved a formal characterization for the \emph{Catalan} array
    $\mathcal{C}$. These results were presented in a talk contributed at a recent
    conference held in Lecco and is the topic of a submitted paper.

    Currently, he is working with prof. Donatella Merlini
    on advanced topic involving Riordan arrays: in particular, on binary words avoiding patterns, 
    lattice paths enumeration problems and transformations of infinite sequences of numbers.
    He would deepen his understanding of such topics in order to make them central in his PhD thesis;
    moreover, he is designing a symbolic framework to implement a subset of most important and useful
    definitions taken from literature on this field.

    \subsection{Papers submitted and in preparation}

    \begin{itemize}

        \item Donatella Merlini, Massimo Nocentini. \emph{Colouring Catalan triangle}, submitted.

        \item Donatella Merlini, Massimo Nocentini. \emph{Algebraic generating functions for languages
            avoiding Riordan patterns}, in preparation.

        \item Donatella Merlini, Massimo Nocentini. \emph{Patterns in Riordan arrays}\footnote{\url{http://www1.mate.polimi.it/~munarini/RART2015/Abstracts/RART2015_Nocentini.pdf}}, 
            Second International Symposium on Riordan Arrays and Related Topics, Lecco, 2015.

        \item \emph{Recurrence unfolding}\footnote{\url{https://github.com/massimo-nocentini/recurrences-unfolding}}: 
            We provide a framework, written using the Python language
            on top of \texttt{SymPy} module, to perform arbitrary unfolding of recurrence relations. The main idea
            is to consider a set of possibly mutually defined recurrence relations, call it $\Omega$, and use each
            one of them as a \emph{rewriting rule} in the sense of using the left-hand side (``lhs'' for short) 
            as a term to be matched in order to instantiate the right-hand side (``rhs'', respectively) accordingly; 
            finally, use the new rhs as a replacement for the term that starts the matching. 

            Unlike "plain" substitution, we perform an extended matching strategy on the lhs, 
            allowing the mathematician to write relations that include a coefficient in the lhs: so a term in the rhs
            matches successful if it is possible to find a substitution for free variables that makes equal both
            the indexed symbol and the coefficient.

            To the time of this document, we have a working prototype for relations that involve indexed terms of the
            form $f_{n_{1}, \ldots, n_{k}}$ for desired $k\in\mathbb{N}$, with the constraint that the recurrence relation 
            use constant coefficients. We're working to fully handle arbitrary recurrence relations. For the sake of 
            clarity, we show, first, an application to the sequence of Fibonacci numbers\footnote{\url{http://nbviewer.jupyter.org/github/massimo-nocentini/recurrences-unfolding/blob/master/notebooks/fibonacci-numbers-unary-indexed-unfolding.ipynb}},
            according to the unary-indexed recurrence $f_{n+2}=f_{n+1}+f_{n}$; second, an application to the Pascal array
            \footnote{\url{http://nbviewer.jupyter.org/github/massimo-nocentini/recurrences-unfolding/blob/master/notebooks/pascal-array-doubly-indexed-unfolding.ipynb}},
            according to the doubly-indexed recurrence $d_{n+1,k+1} = d_{n,k}+d_{n,k+1}$.

            We aim to show possibly new or hard to recognize identities over classes of combinatorial objects counted by
            relations under study; therefore, this prototype could be seen as an helper for the mathematician to understand
            how a recurrence behaves doing unfolding, leaving to him/her the analytic check of spotted patterns seen 
            while unfolding the recurrence. In preparation.

        \item \emph{Riordan Arrays}: Following ideas of unfolding recurrence relations, we consider matrices in the Riordan group,
            leaving them completely symbolical, focusing only on their $A$-sequences which dictates relations among coefficients.
            We choose a row $r$ and start to unfold coefficients from row $r+1$ to the last, in order to build a matrix that 
            depends on a small set of coefficients, namely those lying on the former $r+1$ rows. This allow us to rewrite
            the original matrix as a sum of ``inner'' matrices, all of them sharing the same relation among coefficients.
            We're working on a prototype that built such splitting, collecting material in a notebook
            \footnote{\url{http://nbviewer.jupyter.org/github/massimo-nocentini/recurrences-unfolding/blob/master/notebooks/matrix-recurrences-unfolding.ipynb}}.
            In preparation.

        \item\emph{OEIS mining}: We develop a \emph{web-crawler} targeting the OEIS \footnote{\url{http://oeis.org/}}, in order
            to fetch sequences according to the \emph{cross-references} section included in search results. Collected 
            informations can be represented as a graph, where each node represent a sequence and there is a reference
            between two sequences if they appears in the cited section. Such graph can be manipulated in order to remove
            not interesting sequences for the study of interest, in order to apply graph mining techniques. This is an
            ongoing project, however we have a first application\footnote{\url{http://nbviewer.jupyter.org/github/massimo-nocentini/competitive-programming/blob/master/tutorials/oeis-mining.ipynb?flush_cache=true}}. In preparation.

        \item\emph{OEIS search result pretty printer}: On top of the web-crawler defined in the previous topic, 
            we provide a bunch of Python functions to implement a
            \emph{pretty printer} for search result returned by OEIS. Use Jupyter notebook as working
            environments, we provide an API in order to query the OEIS by either sequence identifier or
            segment or open content, parsing results as json documents and returning an IPython display object
            that shows them directly in your notebook. This approach has the following ideas: simple API to 
            perform a search, filtering all sections provided by the OEIS, list and table representation 
            for sequences that have one and two dimensions, respectively. We believe that having an integrated
            environment, without neither switching between browser tabs nor put references to external contents, 
            allows us to build reproducible, self-contained notebooks. We provide a tutorial\footnote{\url{http://nbviewer.jupyter.org/github/massimo-nocentini/competitive-programming/blob/master/tutorials/oeis-interaction.ipynb?flush_cache=true}}. In preparation.

    \end{itemize}
    
    
    \subsection{Conferences}

    \begin{itemize}
        \item \emph{ECOOP}\footnote{\url{http://2016.ecoop.org/}}, July 2016 Rome, Italy: attended as volunteer student.
        \item \emph{Second International Symposium on Riordan Arrays and Related Topics}\footnote{\url{https://www.mate.polimi.it/RART2015/}}, 
            July 2015 Lecco, Italy: contributed talk about modular Catalan triangle $\mathcal{C}_{\equiv_{2}}$
    \end{itemize}

    \subsection{Seminars}

    Given a summary of my first year PhD's activities, the talk is available on-line\footnote{\url{http://massimo-nocentini.github.io/PhD/first-year-summary/talk.html#/}}.

    \subsection{Teaching}

    Given a class about \emph{SymPy}\footnote{\url{http://www.sympy.org/en/index.html}} to introduce
    symbolic manipulations on top of the Python language, within a course on \emph{Analysis of Algorithms}
    taught by Donatella Merlini at the University of Florence. In addition, we translate lab sessions using
    SymPy objects in notebooks, freely available\footnote{\url{https://github.com/massimo-nocentini/pacc/tree/master/paa-course}}.

    \section{Working activity}

    During his studies he worked in middle-size software houses developing mainly client-server
    applications using industrial-strength languages such as Java and C\#, for about eight years.
        
    
\end{document}
