% \documentclass[totpages,helvetica,openbib,italian]{europecv}
\documentclass[totpages,openbib,italian]{europecv}
\usepackage[T1]{fontenc}
\usepackage{graphicx}
\usepackage[a4paper,top=1.27cm,left=1cm,right=1cm,bottom=2cm]{geometry}
\usepackage[italian]{babel}
\usepackage{bibentry}
\usepackage{url}
\usepackage{charter}
%\usepackage[nochapters,beramono,eulermath]{classicthesis}

%\renewcommand{\ttdefault}{phv} % Uses Helvetica instead of fixed width font

\ecvname{Nocentini Massimo}
%\ecvfootername{Nome/i Cognome/i}
\ecvaddress{Via Ronco Corto 98, 50143, Firenze, Italia}
\ecvtelephone{+39 3201162059}
\ecvemail{\url{massimo.nocentini@gmail.com}\quad\url{massimo.nocentini@unifi.it}}
\ecvnationality{Italiana}
\ecvdateofbirth{08/01/1986}
\ecvgender{Maschile}
%\ecvbeforepicture{\raggedleft}
\ecvpicture[width=5cm]{me.jpg}
%\ecvafterpicture{\ecvspace{-3.5cm}}
\ecvfootnote{Per ulteriori informazioni:
  \url{https://github.com/massimo-nocentini}}

\begin{document}
\selectlanguage{italian}


\begin{europecv}
\ecvpersonalinfo[20pt]

\ecvitem{\large\textbf{Impiego ricercato/ Settore di
    competenza}}{\large\textbf{Ricercatore, Sviluppatore}}

\ecvsection{Esperienza accademica}

\ecvitem{Date}{\textbf{da 01/11/2015 a 31/10/2018}}
\ecvitem{Evento}{PhD Student in Mathematics, Computer Science and Statistics}
\ecvitem{Principali mansioni e responsabilit\`a}
{Percorso di dottorato di ricerca, il mio curriculum studiorum pu\`o essere letto all'indirizzo
\url{https://github.com/massimo-nocentini/curriculum-vitae/releases/download/studiorum-v1.1/cv-studiorum.pdf}}
\ecvitem{Universit\`a}
{Universit\`a degli Studi di Firenze: Dipartimento di Matematica, Informatica e Statistica}
\\
\ecvitem{Date}{\textbf{da 01/07/2013 a 01/01/2015  }}
\ecvitem{Funzione o posto occupato}{Borsista}
\ecvitem{Principali mansioni e responsabilit\`a}
{Borsa di studio post-laurea \emph{Architetture e metodi
 per la cooperazione applicativa} sotto la supervisione del
 Prof. \emph{Enrico Vicario}: implementazione di un simulatore
 di reti gas, per il calcolo del bilancio di pressione ai nodi
 e bilancio delle portate sui rami. Al rinnovo della borsa,
 nel secondo anno il motore \`e stato esteso per la gestione
 di reti acqua, introducendo nuove features e tipologie di oggetti.
 L'analisi fluido-dinamica \`e stata studiata dal Dott. \emph{Fabio Tarani}, 
 io ho curato l'implementazione usando il linguaggio \texttt{C\#},
 tutto il codice \`e disponibile sotto licenza MIT nel seguente
 repository GIT: \url{git@github.com:massimo-nocentini/network-reasoner.git} }
\ecvitem{Universit\`a}
{Universit\`a degli Studi di Firenze, scuola di Ingegneria}

\ecvsection{Esperienza professionale}

\ecvitem{Date}{\textbf{da 01/01/2019}}
\ecvitem{Funzione o posto occupato}{Libero professionista}
\ecvitem{Principali mansioni e responsabilit\`a}
{Sviluppo su piattaforma Pharo Smalltalk}
\ecvitem{Tipo o settore d'attivit\`a}
{Azienda che opera nel settore terziario e dei servizi}
\\
\ecvitem{Date}{\textbf{da 01/02/2013 a 30/06/2013}}
\ecvitem{Funzione o posto occupato}{Sviluppatore}
\ecvitem{Principali mansioni e responsabilit\`a}
{Sviluppo di un layer middleware per il controllo di sensori e 
    attuatori: demone sui Raspberry scritto in \texttt{C\#} su
    piattaforma \emph{Mono}, ogni dispositivo si interfaccia
    tramite protocollo XMPP; frontend interfaccia web, uso di 
    eventi asincroni, basato su JQuery}
\ecvitem{Datore di lavoro}
{Negens Srl, Via Caravaggio 9, 50100 Firenze, Italia}
\ecvitem{Tipo o settore d'attivit\`a}
{Azienda che opera nel settore terziario e dei servizi}
\\
\ecvitem{Date}{\textbf{da 01/10/2007 a 31/12/2012}}
\ecvitem{Funzione o posto occupato}{Sviluppatore}
\ecvitem{Principali mansioni e responsabilit\`a}
{Sviluppo di applicazioni utilizzando la piattaforma \emph{.NET}, 
  sia basate su architetture standalone e client-server. Alcune applicazioni
  sviluppate: manutenzione, refactoring e sviluppo di un generatore di sistemi
  per giochi orientati agli eventi (schedine sportive, totocalcio), interfacciamento
  via web-services con server Sisal; refactoring e sviluppo di un gestionale 
  per la parte economica, in particolare memorizzazione e manipolazione
  di fatture, registrazioni in partita doppia, compilazione del libro giornale
  e di registri IVA.}
\ecvitem{Datore di lavoro}
{Comm.it, P.IVA 05766090483, Sede legale: Via Mario de Bernardi, 65,
  50145, Firenze (FI), Italy. Sede operativa: Via Antonio Gramsci,
  426,
  50019, Sesto Fiorentino (FI), Italy }
\ecvitem{Tipo o settore d'attivit\`a}
{Azienda che opera nel settore terziario e dei servizi}
\\
\ecvitem{Date}{\textbf{da 03/10/2005 a 03/03/2006}}
\ecvitem{Funzione o posto occupato}{Sviluppatore}
\ecvitem{Principali mansioni}
{Progettazione e sviluppo di tre siti web e assistenza tecnica su
  macchine linux}
\ecvitem{Datore di lavoro}
{AnemoneLab, Via Umberto Giordano 18, 50018 Scandicci, Italia}
\ecvitem{Tipo o settore d'attivit\`a}
{Azienda che opera nel settore terziario e dei servizi}
\\
\ecvitem{Date}{\textbf{da 15/07/2005 a 14/09/2005}}
\ecvitem{Funzione o posto occupato}{Sviluppatore}
\ecvitem{Principali mansioni e responsabilit\`a}
{Realizzazione sito web per la gestione delle attivit\`a di produzione
  e simulazione delle variazioni dei costi, sito per l'intranet locale}
\ecvitem{Datore di lavoro}
{Balena S.R.L., Via de Vespucci 210, 50145 Firenze, Italia}
\ecvitem{Tipo o settore d'attivit\`a}
{Azienda che opera nel settore alimentare}
\\
\ecvitem{Date}{\textbf{da 04/06/2005 a 12/07/2005}}
\ecvitem{Funzione o posto occupato}{Stagista}
\ecvitem{Principali mansioni e responsabilit\`a}
{Stesura del manuale utente di un gestionale sviluppato dall'azienda,
  sia in forma cartacea che digitale, traduzione in lingua inglese,
  analisi e gestione della base di dati del gestionale}
\ecvitem{Datore di lavoro}
{Exitech, Via delle Mantellate 16, 50129 Firenze, Italia}
\ecvitem{Tipo o settore d'attivit\`a}
{Azienda che opera nel settore terziario e dei servizi}
\\
\ecvitem{Date}{\textbf{da 02/09/2004 a 07/01/2005}}
\ecvitem{Funzione o posto occupato}{Sviluppatore}
\ecvitem{Principali mansioni e responsabilit\`a}
{Progettazione e sviluppo di un gestionale che permette di gestire sia
  il reparto produzione che quello economico}
\ecvitem{Datore di lavoro}
{Balena S.R.L., Via de Vespucci 210, 50145 Firenze, Italia}
\ecvitem{Tipo o settore d'attivit\`a}
{Azienda che opera nel settore alimentare}
\\
\ecvitem{Date}{\textbf{da 14/06/2004 a 18/07/2004}}
\ecvitem{Funzione o posto occupato}{Stagista}
\ecvitem{Principali mansioni e responsabilit\`a}
{Realizzazione della basi per il progetto della gestione delle assenze
  e dei ritardi in modo automatico con lettore di codici a barre}
\ecvitem{Datore di lavoro}
{I.T.I.S Antonio Meucci, Via di Scandicci 151, 50137 Firenze, Italia}
\ecvitem{Tipo o settore d'attivit\`a}
{Istituto Scolastico Superiore}

\ecvsection{Istruzione e formazione}

\ecvitem{Date}{\textbf{dal 01/10/2012}}
\ecvitem{Certificato o diploma}{Laureato in C.d.L. MAGISTRALE in INFORMATICA}
\ecvitem{Data discussione}{09/10/2015}
\ecvitem{Votazione}{110 e lode}
\ecvitem{Titolo della tesi}{Patterns in Riordan Arrays}
\ecvitem{Relatore}{Prof. Donatella Merlini}
\ecvitem{Principali materie apprese}{theoretical computer science, discrete mathematics}
\ecvitem{Nome e tipo d'Istituto}{
    Universit\`a degli Studi di Firenze, scuola di Scienze Matematiche Fisiche e Naturali}
\\
\ecvitem{Date}{\textbf{dal 01/10/2005 al 22/02/2012}}
\ecvitem{Certificato o diploma ottenuto}{Laureato in INFORMATICA}
\ecvitem{Data conseguimento}{22/02/2012}
\ecvitem{Votazione}{110}
\ecvitem{Titolo della tesi}{Analisi di reti metaboliche basata su propriet\`a di connessione}
\ecvitem{Relatore}{Prof. Pierluigi Crescenzi}
\ecvitem{Principali materie apprese}{theoretical computer science, discrete mathematics}
\ecvitem{Nome e tipo d'Istituto}{
    Universit\`a degli Studi di Firenze, scuola di Scienze Matematiche Fisiche e Naturali}
\\
\ecvitem{Date}{\textbf{dal 04/09/2000 al 06/07/2005}}
\ecvitem{Certificato o diploma ottenuto}{TECN. INDUSTR. INFORMATICA (durata  5 anni), progetto ABACUS}
\ecvitem{Votazione}{100}
\ecvitem{Principali materie apprese}{Informatica, matematica, sistemi, elettronica}
\ecvitem{Nome e tipo d'Istituto}{
    Istituto Tecnico Industriale \emph{Antonio Meucci}}

\ecvsection{Capacit\`a e competenze professionali}

\ecvmothertongue[10pt]{Italiana}
\ecvitem{\large Altra/e lingua/e}{Inglese}
\ecvlanguageheader{(*)}
\ecvlanguage{Inglese}{\ecvBTwo}{\ecvCOne}{\ecvBTwo}{\ecvBTwo}{\ecvCOne}
\ecvlanguagefooter[10pt]{(*)}

\ecvitem[10pt]{\large Capacit\`a e competenze sociali}{Sto imparando a
raggiungere obiettivi, aiutandosi reciprocamente, nella pratica del \emph{Judo}.
Occasionalmente presto servizio volontario presso il \emph{CUI - Ragazzi del Sole}
un'associazione che aiuta persone disabili}

\ecvitem[10pt]{\large Capacit\`a e competenze organizzative}{Durante
  il periodo di lavoro in Comm.it ho appreso molto riguardo
  l'organizzazione dei processi di produzione software. Sono stato
  affiancato da persone che hanno lavorato nei laboratori
  dell'Universit\`a di Firenze che mi hanno insegnato la loro
  metodologia di lavoro. Abbiamo sempre lavorato con pi\`u di una
  persona per progetto, questo \`e stato molto utile per capire i
  problemi sia di natura tecnica che umana, come poterli affrontare e
  risolvere
}

\ecvitem[10pt]{\large Capacit\`a e competenze tecniche}{La maggior
  parte dei progetti a cui ho participato durante il periodo lavorativo
  hanno avuto come target la piattaforma .NET. 
  Le applicazioni sono state scritte usando il
  linguaggio \texttt{C\#}, inoltre molti di questi progetti
  si sono avvalsi di database relazionali: \`e stato possibile studiare
  sia il linguaggio TSQL ed in maggior misura \emph{PostgreSQL}, 
  affrontando costrutti complessi come il pivoting e manipolazione di query ricorsive.

  Durante la stesura della mia tesi triennale e durante gli anni universitari ho
  studiato il linguaggio Java, utilizzandolo per l'implementazione di
  progetti come argomento d'esame e del codice della mia tesi.

  Fuori dall'ambiente lavorativo il mio interesse \`e rivolto su linguaggi
  funzionali (\emph{Standard ML}, \emph{OCaml}, \emph{CommonLisp}) e 
  logici (\emph{Prolog}): in particolare ho implementato il contenuto
  della serie \emph{Little books}, di Dan Friedman e Matthew Fellaisen,
  portando il loro codice scritto in \emph{Scheme} in una versione
  scritta in \emph{Standard ML}, tutto il progetto \`e disponibile con licenza MIT
  nel repository GIT: \url{git@github.com:massimo-nocentini/reasoning-about-little-books.git}
  (di particolare interesse la parte relativa alla \emph{continuazioni},
  discussa nel volume \emph{The Seasoned Schemer}). Infine, studio 
  nel tempo libero le idee e i principi dei linguaggi orientati
  agli oggetti \emph{Python} e, in particolar modo, \emph{Smalltalk}
}

\ecvitem[10pt]{\large Capacit\`a e competenze informatiche}{ Durante i
  miei studi ho trovato molto interesse riguardo l'Informatica
  Teorica. In particolare gli argomenti che preferisco sono la teoria
  della complessit\`a, dei compilatori e della logica. 
  In parte sto cercando di approcciare
  i primi studiando il linguaggio simbolico LISP, per i secondi sto
  studiando i libri di Raymond Smullyan. Durante lo sviluppo
  della tesi magistrale ho trovato interesse nell'interpretazione combinatoria
  di classi di oggetti, in particolare di una loro manipolazione algebrica
  usando la teoria dei \emph{Riordan arrays}}

\ecvitem[10pt]{\large Capacit\`a e competenze artistiche}{
    Studi di finali di scacchi e del gioco del \emph{Go}}

\ecvitem{\large Patente/i}{Patente di guida B}

\ecvsection{Ulteriori informazioni}
%\ecvitem[10pt]{}{Il prof. Pierluigi Crescenzi mi ha fatto crescere
  %molto e  ci tengo a ringraziarlo. Un suo riferimento \`e \url{http://piluc.dsi.unifi.it/piluc/}}
\bibliographystyle{plain}
\nobibliography{publications}
\ecvitem{}{\textbf{Open-source projects}}
\ecvitem{}{Vedere \url{https://github.com/massimo-nocentini}}
\ecvitem{}{\textbf{Interessi personali}}
\ecvitem{}{Sono interessato agli argomenti trattati in informatica
  teorica, matematica discreta e alla pratica del \emph{Judo}}

\ecvsection{Liberatoria}
\ecvitem[10pt]{}{Autorizzo al trattamento dei dati personali secondo
  quanto previsto dalla legge numero 196/03}

\end{europecv}


\end{document} 
